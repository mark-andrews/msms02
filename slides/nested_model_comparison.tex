% Options for packages loaded elsewhere
\PassOptionsToPackage{unicode}{hyperref}
\PassOptionsToPackage{hyphens}{url}
%
\documentclass[
  10pt,
  ignorenonframetext,
]{beamer}
\title{Nested Model Comparison in General and Generalized Linear Models}
\author{Mark Andrews\\
Psychology Department, Nottingham Trent University\\
\strut \\
\texttt{mark.andrews@ntu.ac.uk}}
\date{}

\usepackage{pgfpages}
\setbeamertemplate{caption}[numbered]
\setbeamertemplate{caption label separator}{: }
\setbeamercolor{caption name}{fg=normal text.fg}
\beamertemplatenavigationsymbolsempty
% Prevent slide breaks in the middle of a paragraph
\widowpenalties 1 10000
\raggedbottom
\setbeamertemplate{part page}{
  \centering
  \begin{beamercolorbox}[sep=16pt,center]{part title}
    \usebeamerfont{part title}\insertpart\par
  \end{beamercolorbox}
}
\setbeamertemplate{section page}{
  \centering
  \begin{beamercolorbox}[sep=12pt,center]{part title}
    \usebeamerfont{section title}\insertsection\par
  \end{beamercolorbox}
}
\setbeamertemplate{subsection page}{
  \centering
  \begin{beamercolorbox}[sep=8pt,center]{part title}
    \usebeamerfont{subsection title}\insertsubsection\par
  \end{beamercolorbox}
}
\AtBeginPart{
  \frame{\partpage}
}
\AtBeginSection{
  \ifbibliography
  \else
    \frame{\sectionpage}
  \fi
}
\AtBeginSubsection{
  \frame{\subsectionpage}
}
\usepackage{amsmath,amssymb}
\usepackage{lmodern}
\usepackage{iftex}
\ifPDFTeX
  \usepackage[T1]{fontenc}
  \usepackage[utf8]{inputenc}
  \usepackage{textcomp} % provide euro and other symbols
\else % if luatex or xetex
  \usepackage{unicode-math}
  \defaultfontfeatures{Scale=MatchLowercase}
  \defaultfontfeatures[\rmfamily]{Ligatures=TeX,Scale=1}
\fi
\usefonttheme{serif}
% Use upquote if available, for straight quotes in verbatim environments
\IfFileExists{upquote.sty}{\usepackage{upquote}}{}
\IfFileExists{microtype.sty}{% use microtype if available
  \usepackage[]{microtype}
  \UseMicrotypeSet[protrusion]{basicmath} % disable protrusion for tt fonts
}{}
\makeatletter
\@ifundefined{KOMAClassName}{% if non-KOMA class
  \IfFileExists{parskip.sty}{%
    \usepackage{parskip}
  }{% else
    \setlength{\parindent}{0pt}
    \setlength{\parskip}{6pt plus 2pt minus 1pt}}
}{% if KOMA class
  \KOMAoptions{parskip=half}}
\makeatother
\usepackage{xcolor}
\IfFileExists{xurl.sty}{\usepackage{xurl}}{} % add URL line breaks if available
\IfFileExists{bookmark.sty}{\usepackage{bookmark}}{\usepackage{hyperref}}
\hypersetup{
  pdftitle={Nested Model Comparison in General and Generalized Linear Models},
  hidelinks,
  pdfcreator={LaTeX via pandoc}}
\urlstyle{same} % disable monospaced font for URLs
\newif\ifbibliography
\usepackage{color}
\usepackage{fancyvrb}
\newcommand{\VerbBar}{|}
\newcommand{\VERB}{\Verb[commandchars=\\\{\}]}
\DefineVerbatimEnvironment{Highlighting}{Verbatim}{commandchars=\\\{\}}
% Add ',fontsize=\small' for more characters per line
\usepackage{framed}
\definecolor{shadecolor}{RGB}{248,248,248}
\newenvironment{Shaded}{\begin{snugshade}}{\end{snugshade}}
\newcommand{\AlertTok}[1]{\textcolor[rgb]{0.94,0.16,0.16}{#1}}
\newcommand{\AnnotationTok}[1]{\textcolor[rgb]{0.56,0.35,0.01}{\textbf{\textit{#1}}}}
\newcommand{\AttributeTok}[1]{\textcolor[rgb]{0.77,0.63,0.00}{#1}}
\newcommand{\BaseNTok}[1]{\textcolor[rgb]{0.00,0.00,0.81}{#1}}
\newcommand{\BuiltInTok}[1]{#1}
\newcommand{\CharTok}[1]{\textcolor[rgb]{0.31,0.60,0.02}{#1}}
\newcommand{\CommentTok}[1]{\textcolor[rgb]{0.56,0.35,0.01}{\textit{#1}}}
\newcommand{\CommentVarTok}[1]{\textcolor[rgb]{0.56,0.35,0.01}{\textbf{\textit{#1}}}}
\newcommand{\ConstantTok}[1]{\textcolor[rgb]{0.00,0.00,0.00}{#1}}
\newcommand{\ControlFlowTok}[1]{\textcolor[rgb]{0.13,0.29,0.53}{\textbf{#1}}}
\newcommand{\DataTypeTok}[1]{\textcolor[rgb]{0.13,0.29,0.53}{#1}}
\newcommand{\DecValTok}[1]{\textcolor[rgb]{0.00,0.00,0.81}{#1}}
\newcommand{\DocumentationTok}[1]{\textcolor[rgb]{0.56,0.35,0.01}{\textbf{\textit{#1}}}}
\newcommand{\ErrorTok}[1]{\textcolor[rgb]{0.64,0.00,0.00}{\textbf{#1}}}
\newcommand{\ExtensionTok}[1]{#1}
\newcommand{\FloatTok}[1]{\textcolor[rgb]{0.00,0.00,0.81}{#1}}
\newcommand{\FunctionTok}[1]{\textcolor[rgb]{0.00,0.00,0.00}{#1}}
\newcommand{\ImportTok}[1]{#1}
\newcommand{\InformationTok}[1]{\textcolor[rgb]{0.56,0.35,0.01}{\textbf{\textit{#1}}}}
\newcommand{\KeywordTok}[1]{\textcolor[rgb]{0.13,0.29,0.53}{\textbf{#1}}}
\newcommand{\NormalTok}[1]{#1}
\newcommand{\OperatorTok}[1]{\textcolor[rgb]{0.81,0.36,0.00}{\textbf{#1}}}
\newcommand{\OtherTok}[1]{\textcolor[rgb]{0.56,0.35,0.01}{#1}}
\newcommand{\PreprocessorTok}[1]{\textcolor[rgb]{0.56,0.35,0.01}{\textit{#1}}}
\newcommand{\RegionMarkerTok}[1]{#1}
\newcommand{\SpecialCharTok}[1]{\textcolor[rgb]{0.00,0.00,0.00}{#1}}
\newcommand{\SpecialStringTok}[1]{\textcolor[rgb]{0.31,0.60,0.02}{#1}}
\newcommand{\StringTok}[1]{\textcolor[rgb]{0.31,0.60,0.02}{#1}}
\newcommand{\VariableTok}[1]{\textcolor[rgb]{0.00,0.00,0.00}{#1}}
\newcommand{\VerbatimStringTok}[1]{\textcolor[rgb]{0.31,0.60,0.02}{#1}}
\newcommand{\WarningTok}[1]{\textcolor[rgb]{0.56,0.35,0.01}{\textbf{\textit{#1}}}}
\setlength{\emergencystretch}{3em} % prevent overfull lines
\providecommand{\tightlist}{%
  \setlength{\itemsep}{0pt}\setlength{\parskip}{0pt}}
\setcounter{secnumdepth}{-\maxdimen} % remove section numbering
\RequirePackage{pifont,manfnt}
\RequirePackage{booktabs}
\usepackage{subcaption}
\RequirePackage[T1]{fontenc}
\RequirePackage{mathpazo}
\RequirePackage{eulervm}
\linespread{1.05}
\RequirePackage{tikz}
\usepackage{pgfplots}
\usetikzlibrary{arrows,positioning,matrix} 
\RequirePackage{xspace}
\RequirePackage{apacite}
\RequirePackage{rotating}
\RequirePackage{multirow}
\usepackage{fontawesome}
\usepackage{nth}
\pgfplotsset{compat=1.16}
\newcommand{\Prob}[1]{\mathrm{P}( #1 )}
\newcommand{\dcat}[1]{\mathrm{dcat}( #1 )}
\newcommand{\ddirichlet}[1]{\mathrm{ddirichlet}( #1 )}
\newcommand*{\given}{\vert}
\newcommand{\hdpmm}{\textsc{hdptm}\xspace}
\newcommand{\bnc}{\textsc{bnc}\xspace}
\newcommand{\brms}{Brms\xspace}
\newcommand{\mcmc}{\textsc{mcmc}\xspace}
\newcommand{\icc}{\textsc{icc}\xspace}
\newcommand{\reml}{\textsc{reml}\xspace}
\newcommand{\mad}{\textsc{mad}\xspace}

\newcommand\iidsim{\mathrel{\overset{\makebox[0pt]{\mbox{\normalfont\tiny iid}}}{\sim}}}
\newcommand\defeq{\mathrel{\overset{\makebox[0pt]{\mbox{\normalfont\tiny def}}}{=}}}
\newcommand{\hpd}{\textsc{hpd}\xspace}
\newcommand{\Probc}[1]{\mathrm{P}_{\text{\!\tiny \textsc{c}}}( #1 )}
\newcommand{\Proba}[1]{\mathrm{P}_{\text{\!\tiny \textsc{a}}}( #1 )}
\newcommand{\wnew}{w_{j}}
\newcommand{\wjinew}{w_{ji}}
\newcommand{\pinew}{\pi_{j}}
\newcommand{\data}{\mathcal{D}}
\newcommand{\dic}{\textsc{dic}\xspace}
\newcommand{\studentt}[1]{t_{#1}}
\setbeamerfont{title}{family=\it}
\setbeamerfont{frametitle}{family=\it}

\RequirePackage{tikz}
\usetikzlibrary{trees}
\usetikzlibrary{matrix}

\RequirePackage{amssymb,latexsym,amsmath,amsfonts,amscd}

\usecolortheme[named=gray]{structure} 
\setbeamercolor{titlelike}{fg=black!60!red}
\definecolor{Mygrey}{gray}{0.75}

\newcommand{\rreallytiny}{\fontsize{3}{3}\selectfont}
\newcommand{\reallytiny}{\fontsize{5}{5}\selectfont}

\usetikzlibrary{decorations.pathmorphing} % noisy shapes
\usetikzlibrary{fit}					% fitting shapes to coordinates
\usetikzlibrary{backgrounds}	% drawing the background after the foreground
\usetikzlibrary{matrix}

\tikzstyle{background}=[rectangle, fill=none,
						draw=black,
                                                inner sep=0.3cm,
                                                rounded corners=3mm]

\tikzstyle{observation}=[circle,font=\small,minimum size=5mm,inner sep=0mm,
                                    draw=black!70,
                                    fill=black!10]

\tikzstyle{state}=[circle,font=\small,minimum size=5mm,inner sep=0mm,
                                   draw=black!70,
                                    fill=none]

\tikzstyle{limit}=[rectangle,font=\small,minimum size=0mm,inner sep=0mm,
                                    fill=none]

\tikzstyle{parameter}=[circle,font=\small,minimum size=5mm,inner sep=0mm,
                                   draw=black!70,
                                    fill=none]
% tikz stuff
\usepackage{tikz}
\usetikzlibrary{shapes}
\usetikzlibrary{positioning,shapes,trees,arrows,shadows,arrows.meta,backgrounds,fit}

\tikzset{every path/.style={-latex,thick}}

\tikzset{
  basic/.style = {font=\sffamily},
  % material/.style  = {basic, text width=8mm, font=\footnotesize\sffamily, fill=yellow!60},
  % revision/.style  = {material, fill=blue!30},
  % root/.style   = {basic,  align=center, fill=pink!60},
  level 1/.style = {basic, align=center, sibling distance = 30mm},
  level 2/.style = {basic, sibling distance = 20mm},
  level 3/.style = {basic, sibling distance = 15mm},
  % level 4/.style = {level distance=10mm,basic, fill=pink!60, sibling distance = 20mm}
}


\graphicspath{{../images/}}

\DeclareSymbolFont{legacymaths}{OT1}{cmr}{m}{n}
\DeclareMathAccent{\dot}     {\mathalpha}{legacymaths}{95}
\DeclareMathAccent{\bar}     {\mathalpha}{legacymaths}{22}
\DeclareMathAccent{\tilde}     {\mathalpha}{legacymaths}{126}

\setbeamertemplate{footline}[frame number]
\ifLuaTeX
  \usepackage{selnolig}  % disable illegal ligatures
\fi

\begin{document}
\frame{\titlepage}

\begin{frame}{What are nested models}
\protect\hypertarget{what-are-nested-models}{}
\begin{itemize}
\tightlist
\item
  Model \(M_0\) is nested in model \(M_1\) if the parameter space of
  \(M_0\) is a subset of the parameter space of \(M_1\).
\item
  For example, if \(M_0\) is the following linear model: \[
  \text{for $i \in 1\ldots n,$} \quad y_i = \beta_0 + \beta_1 x_{1i} + \epsilon_i,\quad \epsilon_i \sim N(0, \sigma^2),
  \] its parameter space is \(\beta_0\), \(\beta_1\), \(\sigma^2\).
\item
  If \(M_1\) is the following linear model: \[
  \text{for $i \in 1\ldots n,$} \quad y_i = \beta_0 + \beta_1 x_{1i} + \beta_2 x_{2i} + \epsilon_i,\quad \epsilon_i \sim N(0, \sigma^2),
  \] its parameter space is \(\beta_0\), \(\beta_1\), \(\beta_2\),
  \(\sigma^2\).
\item
  Any set of values of \(\beta_0\), \(\beta_1\), \(\sigma^2\) in \(M_0\)
  is a point in the parameter space of \(\beta_0\), \(\beta_1\),
  \(\beta_2\), \(\sigma^2\) of \(M_1\) if we simply set \(\beta_2 = 0\).
\item
  In other words, we can make \(M_0\) with any given values of
  \(\beta_0\), \(\beta_1\), \(\sigma^2\) from \(M_1\) by setting
  \(\beta_0\), \(\beta_1\), \(\sigma^2\) in \(M_1\) to these same values
  and setting \(\beta_2 = 0\).
\end{itemize}
\end{frame}

\begin{frame}{Nested normal linear models}
\protect\hypertarget{nested-normal-linear-models}{}
\begin{itemize}
\tightlist
\item
  We can compare nested normal linear models using F tests.
\item
  Assume \(M_0\) and \(M_1\) are normal linear models, with \(M_0\)
  nested in \(M_1\).
\item
  We calculate \(\text{RSS}_0\) and \(\text{RSS}_1\), the residual sums
  of squares of \(M_0\) and \(M_1\), respectively.
\item
  \(\text{RSS}_0\) will be greater than or equal to \(\text{RSS}_1\).
\item
  Then \[
  \begin{aligned}
  \text{proportional increase in error}&= \frac{\text{increase in error}}{\text{minimal error}} ,\\
  \\
  &= \frac{\text{RSS}_0 - \text{RSS}_1}{\text{RSS}_1},
  \end{aligned}
  \]
\end{itemize}
\end{frame}

\begin{frame}[fragile]{Nested normal linear models}
\protect\hypertarget{nested-normal-linear-models-1}{}
\begin{Shaded}
\begin{Highlighting}[]
\NormalTok{M1 }\OtherTok{\textless{}{-}} \FunctionTok{lm}\NormalTok{(Fertility }\SpecialCharTok{\textasciitilde{}}\NormalTok{ Agriculture }\SpecialCharTok{+}\NormalTok{ Education }\SpecialCharTok{+}\NormalTok{ Catholic, }\AttributeTok{data =}\NormalTok{ swiss)}
\NormalTok{M0 }\OtherTok{\textless{}{-}} \FunctionTok{lm}\NormalTok{(Fertility }\SpecialCharTok{\textasciitilde{}}\NormalTok{ Agriculture }\SpecialCharTok{+}\NormalTok{ Education, }\AttributeTok{data =}\NormalTok{ swiss)}

\NormalTok{RSS\_0 }\OtherTok{\textless{}{-}} \FunctionTok{sum}\NormalTok{(}\FunctionTok{residuals}\NormalTok{(M0)}\SpecialCharTok{\^{}}\DecValTok{2}\NormalTok{)}
\NormalTok{RSS\_1 }\OtherTok{\textless{}{-}} \FunctionTok{sum}\NormalTok{(}\FunctionTok{residuals}\NormalTok{(M1)}\SpecialCharTok{\^{}}\DecValTok{2}\NormalTok{)}

\FunctionTok{c}\NormalTok{(RSS\_0, RSS\_1)}
\CommentTok{\#\textgreater{} [1] 3953.270 2567.884}
\end{Highlighting}
\end{Shaded}

\begin{Shaded}
\begin{Highlighting}[]
\NormalTok{(RSS\_0 }\SpecialCharTok{{-}}\NormalTok{ RSS\_1)}\SpecialCharTok{/}\NormalTok{RSS\_1}
\CommentTok{\#\textgreater{} [1] 0.5395049}
\end{Highlighting}
\end{Shaded}

In other words, \(\text{RSS}_0\) is 1.54 greater than \(\text{RSS}_1\).
\end{frame}

\begin{frame}[fragile]{Nested normal linear models}
\protect\hypertarget{nested-normal-linear-models-2}{}
\begin{itemize}
\tightlist
\item
  The F ratio is \[
  \begin{aligned}
  F &= \underbrace{\frac{\text{RSS}_0 - \text{RSS}_1}{\text{RSS}_1}}_{\text{effect size}} \times \underbrace{\frac{\text{df}_1}{\text{df}_0 - \text{df}_1}}_{\text{sample size}}
  &= \frac{(\text{RSS}_0 - \text{RSS}_1)/(\text{df}_0 - \text{df}_1)}{\text{RSS}_1/\text{df}_1}.
  \end{aligned}
  \] where \(\text{df}_1\) is \(N - (K_1 + 1)\), where \(K_1\) is number
  of (predictor; excluding intercept) coefficients in \(M_1\).
\end{itemize}

\begin{Shaded}
\begin{Highlighting}[]
\NormalTok{df\_0 }\OtherTok{\textless{}{-}}\NormalTok{ M0}\SpecialCharTok{$}\NormalTok{df.residual}
\NormalTok{df\_1 }\OtherTok{\textless{}{-}}\NormalTok{ M1}\SpecialCharTok{$}\NormalTok{df.residual}
\FunctionTok{c}\NormalTok{(df\_0, df\_1, df\_0 }\SpecialCharTok{{-}}\NormalTok{ df\_1, df\_1}\SpecialCharTok{/}\NormalTok{(df\_0 }\SpecialCharTok{{-}}\NormalTok{ df\_1))}
\CommentTok{\#\textgreater{} [1] 44 43  1 43}
\end{Highlighting}
\end{Shaded}
\end{frame}

\begin{frame}[fragile]{Nested normal linear models}
\protect\hypertarget{nested-normal-linear-models-3}{}
\begin{Shaded}
\begin{Highlighting}[]
\NormalTok{RSS\_0}
\CommentTok{\#\textgreater{} [1] 3953.27}
\NormalTok{RSS\_1}
\CommentTok{\#\textgreater{} [1] 2567.884}
\NormalTok{RSS\_0 }\SpecialCharTok{{-}}\NormalTok{ RSS\_1}
\CommentTok{\#\textgreater{} [1] 1385.386}
\NormalTok{df\_0 }\SpecialCharTok{{-}}\NormalTok{ df\_1}
\CommentTok{\#\textgreater{} [1] 1}
\NormalTok{df\_1}
\CommentTok{\#\textgreater{} [1] 43}
\NormalTok{(RSS\_0 }\SpecialCharTok{{-}}\NormalTok{ RSS\_1)}\SpecialCharTok{/}\NormalTok{(df\_0 }\SpecialCharTok{{-}}\NormalTok{ df\_1)}
\CommentTok{\#\textgreater{} [1] 1385.386}
\NormalTok{RSS\_1}\SpecialCharTok{/}\NormalTok{df\_1}
\CommentTok{\#\textgreater{} [1] 59.71823}
\NormalTok{((RSS\_0 }\SpecialCharTok{{-}}\NormalTok{ RSS\_1)}\SpecialCharTok{/}\NormalTok{(df\_0 }\SpecialCharTok{{-}}\NormalTok{ df\_1))}\SpecialCharTok{/}\NormalTok{(RSS\_1}\SpecialCharTok{/}\NormalTok{df\_1)}
\CommentTok{\#\textgreater{} [1] 23.19871}
\end{Highlighting}
\end{Shaded}
\end{frame}

\begin{frame}[fragile]{Nested normal linear models}
\protect\hypertarget{nested-normal-linear-models-4}{}
\begin{Shaded}
\begin{Highlighting}[]
\FunctionTok{anova}\NormalTok{(M0, M1)}
\CommentTok{\#\textgreater{} Analysis of Variance Table}
\CommentTok{\#\textgreater{} }
\CommentTok{\#\textgreater{} Model 1: Fertility \textasciitilde{} Agriculture + Education}
\CommentTok{\#\textgreater{} Model 2: Fertility \textasciitilde{} Agriculture + Education + Catholic}
\CommentTok{\#\textgreater{}   Res.Df    RSS Df Sum of Sq      F    Pr(\textgreater{}F)    }
\CommentTok{\#\textgreater{} 1     44 3953.3                                  }
\CommentTok{\#\textgreater{} 2     43 2567.9  1    1385.4 23.199 1.842e{-}05 ***}
\CommentTok{\#\textgreater{} {-}{-}{-}}
\CommentTok{\#\textgreater{} Signif. codes:  0 \textquotesingle{}***\textquotesingle{} 0.001 \textquotesingle{}**\textquotesingle{} 0.01 \textquotesingle{}*\textquotesingle{} 0.05 \textquotesingle{}.\textquotesingle{} 0.1 \textquotesingle{} \textquotesingle{} 1}
\end{Highlighting}
\end{Shaded}
\end{frame}

\begin{frame}[fragile]{Nested normal linear models}
\protect\hypertarget{nested-normal-linear-models-5}{}
\begin{Shaded}
\begin{Highlighting}[]
\FunctionTok{drop1}\NormalTok{(M1, }\AttributeTok{scope =} \SpecialCharTok{\textasciitilde{}}\NormalTok{ Catholic, }\AttributeTok{test =} \StringTok{\textquotesingle{}F\textquotesingle{}}\NormalTok{)}
\CommentTok{\#\textgreater{} Single term deletions}
\CommentTok{\#\textgreater{} }
\CommentTok{\#\textgreater{} Model:}
\CommentTok{\#\textgreater{} Fertility \textasciitilde{} Agriculture + Education + Catholic}
\CommentTok{\#\textgreater{}          Df Sum of Sq    RSS    AIC F value    Pr(\textgreater{}F)    }
\CommentTok{\#\textgreater{} \textless{}none\textgreater{}                2567.9 196.03                      }
\CommentTok{\#\textgreater{} Catholic  1    1385.4 3953.3 214.31  23.199 1.842e{-}05 ***}
\CommentTok{\#\textgreater{} {-}{-}{-}}
\CommentTok{\#\textgreater{} Signif. codes:  0 \textquotesingle{}***\textquotesingle{} 0.001 \textquotesingle{}**\textquotesingle{} 0.01 \textquotesingle{}*\textquotesingle{} 0.05 \textquotesingle{}.\textquotesingle{} 0.1 \textquotesingle{} \textquotesingle{} 1}
\end{Highlighting}
\end{Shaded}
\end{frame}

\begin{frame}[fragile]{Nested normal linear models}
\protect\hypertarget{nested-normal-linear-models-6}{}
\begin{Shaded}
\begin{Highlighting}[]
\FunctionTok{drop1}\NormalTok{(M0, }\AttributeTok{scope =} \SpecialCharTok{\textasciitilde{}}\NormalTok{ Education, }\AttributeTok{test =} \StringTok{\textquotesingle{}F\textquotesingle{}}\NormalTok{)}
\CommentTok{\#\textgreater{} Single term deletions}
\CommentTok{\#\textgreater{} }
\CommentTok{\#\textgreater{} Model:}
\CommentTok{\#\textgreater{} Fertility \textasciitilde{} Agriculture + Education}
\CommentTok{\#\textgreater{}           Df Sum of Sq    RSS    AIC F value    Pr(\textgreater{}F)    }
\CommentTok{\#\textgreater{} \textless{}none\textgreater{}                 3953.3 214.31                      }
\CommentTok{\#\textgreater{} Education  1    2329.8 6283.1 234.09  25.931 7.105e{-}06 ***}
\CommentTok{\#\textgreater{} {-}{-}{-}}
\CommentTok{\#\textgreater{} Signif. codes:  0 \textquotesingle{}***\textquotesingle{} 0.001 \textquotesingle{}**\textquotesingle{} 0.01 \textquotesingle{}*\textquotesingle{} 0.05 \textquotesingle{}.\textquotesingle{} 0.1 \textquotesingle{} \textquotesingle{} 1}
\end{Highlighting}
\end{Shaded}
\end{frame}

\begin{frame}[fragile]{Nested normal linear models}
\protect\hypertarget{nested-normal-linear-models-7}{}
\begin{Shaded}
\begin{Highlighting}[]
\FunctionTok{drop1}\NormalTok{(M0, }\AttributeTok{scope =} \SpecialCharTok{\textasciitilde{}}\NormalTok{ Education }\SpecialCharTok{+}\NormalTok{ Agriculture, }\AttributeTok{test =} \StringTok{\textquotesingle{}F\textquotesingle{}}\NormalTok{)}
\CommentTok{\#\textgreater{} Single term deletions}
\CommentTok{\#\textgreater{} }
\CommentTok{\#\textgreater{} Model:}
\CommentTok{\#\textgreater{} Fertility \textasciitilde{} Agriculture + Education}
\CommentTok{\#\textgreater{}             Df Sum of Sq    RSS    AIC F value    Pr(\textgreater{}F)    }
\CommentTok{\#\textgreater{} \textless{}none\textgreater{}                   3953.3 214.31                      }
\CommentTok{\#\textgreater{} Education    1   2329.85 6283.1 234.09 25.9312 7.105e{-}06 ***}
\CommentTok{\#\textgreater{} Agriculture  1     61.97 4015.2 213.04  0.6897    0.4108    }
\CommentTok{\#\textgreater{} {-}{-}{-}}
\CommentTok{\#\textgreater{} Signif. codes:  0 \textquotesingle{}***\textquotesingle{} 0.001 \textquotesingle{}**\textquotesingle{} 0.01 \textquotesingle{}*\textquotesingle{} 0.05 \textquotesingle{}.\textquotesingle{} 0.1 \textquotesingle{} \textquotesingle{} 1}
\end{Highlighting}
\end{Shaded}
\end{frame}

\begin{frame}[fragile]{Nested normal linear models}
\protect\hypertarget{nested-normal-linear-models-8}{}
\begin{Shaded}
\begin{Highlighting}[]
\FunctionTok{anova}\NormalTok{(M0)}
\CommentTok{\#\textgreater{} Analysis of Variance Table}
\CommentTok{\#\textgreater{} }
\CommentTok{\#\textgreater{} Response: Fertility}
\CommentTok{\#\textgreater{}             Df Sum Sq Mean Sq F value    Pr(\textgreater{}F)    }
\CommentTok{\#\textgreater{} Agriculture  1  894.8  894.84  9.9596  0.002886 ** }
\CommentTok{\#\textgreater{} Education    1 2329.8 2329.85 25.9312 7.105e{-}06 ***}
\CommentTok{\#\textgreater{} Residuals   44 3953.3   89.85                      }
\CommentTok{\#\textgreater{} {-}{-}{-}}
\CommentTok{\#\textgreater{} Signif. codes:  0 \textquotesingle{}***\textquotesingle{} 0.001 \textquotesingle{}**\textquotesingle{} 0.01 \textquotesingle{}*\textquotesingle{} 0.05 \textquotesingle{}.\textquotesingle{} 0.1 \textquotesingle{} \textquotesingle{} 1}
\end{Highlighting}
\end{Shaded}
\end{frame}

\begin{frame}[fragile]{\(R^2\)}
\protect\hypertarget{r2}{}
\begin{itemize}
\tightlist
\item
  If we have two models, \(M_0\) and \(M_1\), with \(M_0\) nested in
  \(M_1\), and with residual sums of squares \(\text{RSS}_0\) and
  \(\text{RSS}_1\), respectively, we can calculate: \[
  \begin{aligned}
  \text{proportional decrease in error}&= \frac{\text{decrease in error (from $M_0$ to $M_1$)}}{\text{error in $M_0$}} ,\\
  \\
  &= \frac{\text{RSS}_0 - \text{RSS}_1}{\text{RSS}_0},\\
  & = R^2
  \end{aligned}
  \]
\end{itemize}

\begin{Shaded}
\begin{Highlighting}[]
\NormalTok{(RSS\_0 }\SpecialCharTok{{-}}\NormalTok{ RSS\_1) }\SpecialCharTok{/}\NormalTok{ RSS\_0}
\CommentTok{\#\textgreater{} [1] 0.3504405}
\end{Highlighting}
\end{Shaded}

\begin{itemize}
\tightlist
\item
  In other words, the reduction in error from \(M_0\) to \(M_1\) is 0.35
  of the error of \(M_0\).
\end{itemize}
\end{frame}

\begin{frame}{\(R^2\): The coefficient of determination}
\protect\hypertarget{r2-the-coefficient-of-determination}{}
\begin{itemize}
\item
  It can be shown that
  \[\underbrace{\sum_{i=1}^n (y_i-\bar{y})^2}_{\text{TSS}} = \underbrace{\sum_{i=1}^n (\hat{y}_i - \bar{y})^2}_{\text{ESS}} + \underbrace{\sum_{i=1}^n (y_i - \hat{y}_i)^2}_{\text{RSS}},\]
  where TSS is \emph{total} sum of squares, ESS is \emph{explained} sum
  of squares, and RSS is \emph{residual} sum of squares.
\item
  The coefficient of determination \(R^2\) is defined as
  \[\begin{aligned}
  R^2 = \frac{\text{\footnotesize ESS}}{\text{\footnotesize TSS}} &= \text{\footnotesize Proportion of variation that is explained},\\
  &= 1 - \frac{\sum_{i=1}^n (y_i - \hat{y}_i)^2 }{\sum_{i=1}^n (y_i-\bar{y})^2 }\end{aligned}\]
\end{itemize}
\end{frame}

\begin{frame}{\(R^2\)}
\protect\hypertarget{r2-1}{}
\begin{itemize}
\item
  If \(M_0\) is a \emph{null} model, i.e.~no predictors, then
  \(\text{TSS}\) = \(\text{RSS}_0\).
\item
  It can be shown that
  \[\underbrace{\sum_{i=1}^n (y_i-\bar{y})^2}_{\text{RSS}_0} = \underbrace{\sum_{i=1}^n (\hat{y}_i - \bar{y})^2}_{\text{RSS}_0 - \text{RSS}_1} + \underbrace{\sum_{i=1}^n (y_i - \hat{y}_i)^2}_{\text{RSS}_1}.\]
\item
  As such, \(R^2\) is defined as \[\begin{aligned}
  R^2 = \frac{\text{RSS}_0-\text{RSS}_1}{\text{RSS}_0} = 1 - \frac{\text{RSS}_1}{\text{RSS}_0}
  \end{aligned},\] or 1 minus the error of \(M_1\) relative to \(M_0\).
\end{itemize}
\end{frame}

\begin{frame}[fragile]{\(R^2\)}
\protect\hypertarget{r2-2}{}
\begin{Shaded}
\begin{Highlighting}[]
\NormalTok{M\_null }\OtherTok{\textless{}{-}} \FunctionTok{lm}\NormalTok{(Fertility }\SpecialCharTok{\textasciitilde{}} \DecValTok{1}\NormalTok{, }\AttributeTok{data =}\NormalTok{ swiss)}
\NormalTok{RSS\_null }\OtherTok{\textless{}{-}} \FunctionTok{sum}\NormalTok{(}\FunctionTok{residuals}\NormalTok{(M\_null)}\SpecialCharTok{\^{}}\DecValTok{2}\NormalTok{)}
\NormalTok{RSS\_0 }\SpecialCharTok{/}\NormalTok{ RSS\_null}
\CommentTok{\#\textgreater{} [1] 0.5507516}
\DecValTok{1} \SpecialCharTok{{-}}\NormalTok{ RSS\_0 }\SpecialCharTok{/}\NormalTok{ RSS\_null}
\CommentTok{\#\textgreater{} [1] 0.4492484}
\NormalTok{(RSS\_null }\SpecialCharTok{{-}}\NormalTok{ RSS\_0) }\SpecialCharTok{/}\NormalTok{ RSS\_null}
\CommentTok{\#\textgreater{} [1] 0.4492484}
\FunctionTok{summary}\NormalTok{(M0)}\SpecialCharTok{$}\NormalTok{r.squared}
\CommentTok{\#\textgreater{} [1] 0.4492484}
\end{Highlighting}
\end{Shaded}
\end{frame}

\begin{frame}{Adjusted \(R^2\)}
\protect\hypertarget{adjusted-r2}{}
\begin{itemize}
\item
  By explaining proportion of variance explained, \(R^2\) is used a
  \emph{goodness of fit} measure.
\item
  However, \(R^2\) will always grow with \(K\), the number of
  predictors.
\item
  \(R^2\) can be \emph{adjusted} to counteract the artificial effect of
  increasing numbers of predictors as follows:
  \[%R^2_{\text{Adj}} = 1 - (1-R^2) \frac{n-1}{n-K-1}
  R^2_{\text{Adj}}  = \underbrace{1-\frac{\text{RSS}}{\text{TSS}}}_{R^2}\underbrace{\frac{n-1}{n-K-1}}_{\text{penalty}},\]
  where \(n\) is sample size.
\item
  \(R^2_{\text{Adj}}\) is not identical to the proportion of variance
  explained in the \emph{sample}, but is an unbiased measured of the
  population \(R^2\).
\end{itemize}
\end{frame}

\begin{frame}[fragile]{Adjusted \(R^2\)}
\protect\hypertarget{adjusted-r2-1}{}
\begin{Shaded}
\begin{Highlighting}[]
\NormalTok{n }\OtherTok{\textless{}{-}} \FunctionTok{nrow}\NormalTok{(M0}\SpecialCharTok{$}\NormalTok{model)}
\NormalTok{K }\OtherTok{\textless{}{-}} \FunctionTok{length}\NormalTok{(}\FunctionTok{coef}\NormalTok{(M0)) }\SpecialCharTok{{-}} \DecValTok{1} \CommentTok{\# no. of predictor coefs}
\NormalTok{penalty }\OtherTok{\textless{}{-}}\NormalTok{ (n }\SpecialCharTok{{-}} \DecValTok{1}\NormalTok{)}\SpecialCharTok{/}\NormalTok{(n }\SpecialCharTok{{-}}\NormalTok{ K }\SpecialCharTok{{-}} \DecValTok{1}\NormalTok{)}
\DecValTok{1} \SpecialCharTok{{-}}\NormalTok{ (RSS\_0 }\SpecialCharTok{/}\NormalTok{ RSS\_null) }\SpecialCharTok{*}\NormalTok{ penalty}
\CommentTok{\#\textgreater{} [1] 0.4242143}
\FunctionTok{summary}\NormalTok{(M0)}\SpecialCharTok{$}\NormalTok{adj.r.squared}
\CommentTok{\#\textgreater{} [1] 0.4242143}
\end{Highlighting}
\end{Shaded}
\end{frame}

\begin{frame}{Deviance}
\protect\hypertarget{deviance}{}
\begin{itemize}
\item
  The \emph{deviance} of a model is defined
  \[-2 \log  L(\hat{\beta}\given\mathcal{D}) ,\] where \(\hat{\beta}\)
  are the mle estimates.
\item
  The better the model fit, the \emph{lower} the deviance.
\item
  This can be seen as equivalent to RSS for generalized linear models.
\end{itemize}
\end{frame}

\begin{frame}[fragile]{Deviance}
\protect\hypertarget{deviance-1}{}
\begin{Shaded}
\begin{Highlighting}[]
\NormalTok{swiss\_df }\OtherTok{\textless{}{-}} \FunctionTok{mutate}\NormalTok{(swiss, }\AttributeTok{y =}\NormalTok{ Fertility }\SpecialCharTok{\textgreater{}} \FunctionTok{median}\NormalTok{(Fertility))}
\NormalTok{M1 }\OtherTok{\textless{}{-}} \FunctionTok{glm}\NormalTok{(y }\SpecialCharTok{\textasciitilde{}}\NormalTok{ Agriculture }\SpecialCharTok{+}\NormalTok{ Education }\SpecialCharTok{+}\NormalTok{ Catholic, }
          \AttributeTok{data =}\NormalTok{ swiss\_df, }\AttributeTok{family =} \FunctionTok{binomial}\NormalTok{())}
\NormalTok{M0 }\OtherTok{\textless{}{-}} \FunctionTok{glm}\NormalTok{(y }\SpecialCharTok{\textasciitilde{}}\NormalTok{ Agriculture }\SpecialCharTok{+}\NormalTok{ Education, }\AttributeTok{data =}\NormalTok{ swiss\_df, }
          \AttributeTok{family =} \FunctionTok{binomial}\NormalTok{())}

\NormalTok{D\_0 }\OtherTok{\textless{}{-}} \FunctionTok{deviance}\NormalTok{(M0)}
\NormalTok{D\_1 }\OtherTok{\textless{}{-}} \FunctionTok{deviance}\NormalTok{(M1)}

\FunctionTok{c}\NormalTok{(D\_0, D\_1)}
\CommentTok{\#\textgreater{} [1] 54.37716 40.70328}
\NormalTok{(D\_0 }\SpecialCharTok{{-}}\NormalTok{ D\_1) }\SpecialCharTok{/}\NormalTok{ D\_1 }\CommentTok{\# prop. incr. error}
\CommentTok{\#\textgreater{} [1] 0.3359405}
\NormalTok{(D\_0 }\SpecialCharTok{{-}}\NormalTok{ D\_1) }\SpecialCharTok{/}\NormalTok{ D\_0 }\CommentTok{\# equiv to R\^{}2?}
\CommentTok{\#\textgreater{} [1] 0.2514637}
\end{Highlighting}
\end{Shaded}
\end{frame}

\begin{frame}{Model comparison with deviance}
\protect\hypertarget{model-comparison-with-deviance}{}
\begin{itemize}
\item
  Let us assume we have two models: \(M_1\) and \(M_0\) where \(M_0\) is
  nested in \(M_1\).
\item
  The deviance of \(M_0\) minus the deviance of the \(M_1\) is
  \[\Delta_{D} = D_0 - D_1.\]
\item
  Under the null hypothesis, \(\Delta_D\) is distributed as \(\chi^2\)
  with \(K_1 - K_0\) df, where \(K_1\) is the number of parameters in
  \(M_1\) and \(K_0\) is the number of parameters in \(M_0\).
\end{itemize}
\end{frame}

\begin{frame}[fragile]{Model comparison with deviance}
\protect\hypertarget{model-comparison-with-deviance-1}{}
\begin{Shaded}
\begin{Highlighting}[]
\NormalTok{K\_0 }\OtherTok{\textless{}{-}} \FunctionTok{length}\NormalTok{(}\FunctionTok{coef}\NormalTok{(M0))}
\NormalTok{K\_1 }\OtherTok{\textless{}{-}} \FunctionTok{length}\NormalTok{(}\FunctionTok{coef}\NormalTok{(M1))}

\FunctionTok{pchisq}\NormalTok{(D\_0 }\SpecialCharTok{{-}}\NormalTok{ D\_1, }
       \AttributeTok{df =}\NormalTok{ K\_1 }\SpecialCharTok{{-}}\NormalTok{ K\_0,}
       \AttributeTok{lower.tail =}\NormalTok{ F)}
\CommentTok{\#\textgreater{} [1] 0.0002174578}
\FunctionTok{anova}\NormalTok{(M0, M1, }\AttributeTok{test =} \StringTok{\textquotesingle{}Chisq\textquotesingle{}}\NormalTok{)}
\CommentTok{\#\textgreater{} Analysis of Deviance Table}
\CommentTok{\#\textgreater{} }
\CommentTok{\#\textgreater{} Model 1: y \textasciitilde{} Agriculture + Education}
\CommentTok{\#\textgreater{} Model 2: y \textasciitilde{} Agriculture + Education + Catholic}
\CommentTok{\#\textgreater{}   Resid. Df Resid. Dev Df Deviance  Pr(\textgreater{}Chi)    }
\CommentTok{\#\textgreater{} 1        44     54.377                          }
\CommentTok{\#\textgreater{} 2        43     40.703  1   13.674 0.0002175 ***}
\CommentTok{\#\textgreater{} {-}{-}{-}}
\CommentTok{\#\textgreater{} Signif. codes:  0 \textquotesingle{}***\textquotesingle{} 0.001 \textquotesingle{}**\textquotesingle{} 0.01 \textquotesingle{}*\textquotesingle{} 0.05 \textquotesingle{}.\textquotesingle{} 0.1 \textquotesingle{} \textquotesingle{} 1}
\end{Highlighting}
\end{Shaded}
\end{frame}

\end{document}
